% !TEX root = ../LinalColloc02.tex
\section{Простейшие дроби. Разложение правильной дроби в сумму простейших}
\begin{conj}
    Простейшая дробь -- любая рациональная дробь вида $\frac{a}{p^n}$, где $p$ -- унитарный неприводимый многочлен, 
    $n$ -- натуральное число, $a$ -- ненулевой многочлен степени меньшей $deg \, p$. 
\end{conj}

\begin{theorem-non}
    Любая ненулевая правильная примарная дробь единственным образом представляется в виде суммы простейших дробей.
\end{theorem-non}

\begin{proof}
    Рассмотрим правильную примарную дробь $\frac{a}{p^n}$. 
    Поделим $a$ на $p$ с отатком: $a = q_1p + r_1$, где $deg \, r_1 < deg \, p$. 
    Тогда $\frac{a}{p^n} = \frac{r_1}{p^n} + \frac{q_1}{p^{n - 1}}$. 
    Теперь поделим $q_1$ на $p$: $q_1 = q_2p + r_2$, где $deg \, r_2 < deg \, p$.
    Тогда $\frac{a}{p^n} = \frac{r_1}{p^n} + \frac{r_2}{p^{n - 1}} + \frac{q_2}{p^{n - 2}}$.
    Продолжая процесс, мы придем к правильной дроби $\frac{q_{n - 1}}{p} = \frac{r_n}{p}$, которая является простейшей. 
    Итак, \[ \frac{a}{p^n} = \frac{r_1}{p^n} + \frac{r_2}{p^{n - 1}} + \dots + \frac{r_n}{p} \]
    Докажем единственность. 
    Предположим, что у нас есть 2 разложения (для удобства поменяем индексы):
    \[ \frac{r_n}{p^n} + \dots + \frac{r_1}{p} = \frac{s_n}{p^n} + \dots + \frac{s_1}{p} \quad \text{(тут некоторые слагаемые могут быть нулевыми)} \]
    Если $m$ -- максимальное значение индекса, при котором $r_m \neq s_m$, то 
    \begin{gather*}
        \frac{r_m - s_m}{p^m} + \dots + \frac{r_1 - s_1}{p} = 0 \\
        s_m - r_m = p(r_{m - 1} - s_{m - 1}) + \dots + p^{m - 1}(r_1 - s_1) = \\
        = p * ((r_{m - 1} - s_{m - 1}) + \dots + p^{m - 2}(r_1 - s_1))
    \end{gather*}
    Мы пришли к противоречию, так как многочлен $s_m - r_m$ имеет степень меньше $deg \, p$, а многочлен справа делит $p$.
\end{proof}

\vspace{7mm}

\textbf{Теорема.}
\textit{Любая ненулевая правильная дробь единственным образом представляется в виде суммы простейших дробей с разными знаменателями.}

\begin{proof}
    Разложим правильную ненулевую дробь $s$ в сумму правильных примарных дробей, разложим каждую примарную в сумму простейших и просуммируем то, что получилось.

    Очевидно, что все знаменатели будут различны, так как для каждой $p$-примарной дроби, знаменатели простейших дробей будут равны $p$ в какой-то степени.
\end{proof}

\vspace{7mm}

\textbf{Пример разложения на простейшие.}

Разложим дробь $\frac{1}{x^5 + x^3}$ над $\mathbb{R}[X]$. 
Разложим знаменатель на неприводимые: $x^5 + x^3 = x^3(x^2 + 1)$.
Выпишем дроби со всеми возможными знаменателями, которые у нас могут получиться:
\[ \frac{1}{x^5 + x^3} = \frac{A}{x} + \frac{B}{x^2} + \frac{C}{x^3} + \frac{Dx + E}{x^2 + 1} \]
Домножим каждую дробь, чтобы избавиться от знаменателей:
\[ 1 = Ax^2(x^2 + 1) + Bx(x^2 + 1) + C(x^2 + 1) + (Dx + E)x^3 \]
Далее воспользуемся методом неопределенных коэффициентов, чтобы найти конкретные $A, \, B, \, C, \, D$ и $E$.
Как говорится, вычисления оставляем читателю.