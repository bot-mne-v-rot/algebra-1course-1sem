% !TEX root = ../LinalColloc02.tex
\section{Примарные дроби. Лемма о дроби, знаменатель которой разложен на два взаимно простых множителя}
\textbf{Лемма.} 
\textit{Если $f$ и $g$ -- взамно простые многочлены, то любую рациональную дробь со знаменателем $fg$
можно представить в виде суммы дробей со знаменателями $f$ и $g$.}

\begin{proof}
    Имеем $1 = cf + dg$ для некототрых $c, \, d \in K[X]$. 
    Но тогда \[ \frac{a}{fg} = \frac{a(cf + dg)}{fg} = \frac{ac}{g} + \frac{ad}{f} \]
\end{proof} 

\begin{conj}
    Примарная дробь -- дробь, которую можно представить в виде $\frac{a}{p^n}$, где $p$ -- неприводимый многочлен, а $n$ -- натуральное число.
\end{conj}
Если нужно указание на конкретный многочлен $p$, то такую дробь называют также $p$-примарной.