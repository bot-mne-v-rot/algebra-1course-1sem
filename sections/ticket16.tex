\section{Определение определителя. Определитель транспонированной матрицы}
\begin{conj}
    $A \in M(n, R), \quad R$ - коммутативное кольцо

    $\abs{A} \in R, \quad A = (a_{ij})$

    $\abs{A} = \det{A} = \sum\limits_{\sigma \in S_n} sgn \ \sigma \cdot \prod\limits_{i=1}^{n} a_{i, \sigma(i)}$

    С первого взгляда какой-то бред, но на самом деле мы просто берем какую-нибудь перестановку $\sigma_i$ из $S_n$, смотрим куда переходит в ней единица,
    берем элемент матрицы $a_{1, \sigma(1)}$, потом также с двойкой и далее пошло поехало. Все это дело перемножаем. Такие телодвижения 
    проделываем для каждой перестановки из $S_n$. И под конец 
    пишем одну огромную сумму из таких произведений, ставя перед ними знак соответствующей перестановки. 
\end{conj}

\textbf{Примеры:}
\begin{itemize}
    \item Определитель матрицы $2 \times 2$:
    
        $\begin{vmatrix}
            a_{11} & a_{12} \\
            a_{21} & a_{22} 
        \end{vmatrix} = a_{11}a_{22} - a_{12}a_{21}$
    \item Определитель матрицы $3 \times 3$:
    
        $\begin{vmatrix}
            a_{11} & a_{12} & a_{13} \\
            a_{21} & a_{22} & a_{23} \\
            a_{31} & a_{32} & a_{33}
        \end{vmatrix} = a_{11}a_{22}a_{33} + a_{12}a_{23}a_{31} + 
        a_{13}a_{21}a_{32} - a_{12}a_{21}a_{33} - a_{13}a_{31}a_{22} - a_{11}a_{23}a_{32}$
\end{itemize}

\begin{theorem-non}
    $\abs{A^T} = \abs{A} \quad A = (a_{ij})$

    $\abs{A^T} = \sum\limits_{\sigma \in S_n} sgn \ \sigma \cdot \prod\limits_{i=1}^{n} a_{\sigma(i), i} =
    \sum\limits_{\sigma \in S_n} sgn \ \sigma \cdot \prod\limits_{i=1}^{n} a_{i, \sigma^{-1}(i)}
    = \sum\limits_{\sigma \in S_n} sgn \ \sigma^{-1} \cdot \prod\limits_{i=1}^{n} a_{i, \sigma(i)} = \abs{A}$
\end{theorem-non}