\section{Определитель матрицы с почти нулевой строкой}

\begin{theorem-non} $\quad$ \\
    $A =
    \begin{pmatrix}
        \dots & \dots & \dots & \dots & \dots & \dots & \dots \\
        0 & \dots & 0 & m & 0 & \dots & 0 \\
        \dots & \dots & \dots & \dots & \dots & \dots & \dots
    \end{pmatrix}, \\
    m$ --- элемент на пересечении $i$-ой строки и $j$-ого столбца.
    
    Тогда
    $|A| = (-1)^{i+j} \cdot m \cdot |M_{ij}|$,
    где $|M_{ij}|$ --- матрица, полученная путём вычёркивания из $A$
    $i$-ой строки и $j$-ого столбца.
    
    $|M_{ij}|$ --- минор, дополненный к элементу $m$.
    
    \begin{proof}
        $1)\; i = j = 1$, т.е.
        $A =
    \begin{pmatrix}
        m & 0 & \dots & 0 \\
        \vdots & & M_{11} &  \\
    \end{pmatrix}$ 
    
        $|A| = m \cdot |M_{11}| = (-1)^{1 + 1} \cdot m \cdot |M_{11}|$
    
        $2)$ Общий случай.
        
        $|A| = (-1)^{i -1} \cdot (-1)^{j - 1} \cdot |A'| = (-1)^{i + j} \cdot m \cdot |M_{ij}|$
    
        $A' = 
        \begin{pmatrix}
            m & 0 & \dots & 0 \\
            \vdots & & M_{ij} & 
        \end{pmatrix}$
    \end{proof}
    \end{theorem-non}