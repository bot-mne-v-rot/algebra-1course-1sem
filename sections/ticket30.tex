% !TEX root = ../LinalColloc02.tex

\section{Равносильные определения базиса}

\begin{conj}
    Пусть $V$ -- ЛП/$K$, $e_1, \dots, e_n \in V$.
    Набор $(e_1, \dots, e_n)$ называется \textbf{базисом линейного 
    пространства} $V$, если $\forall v \in V \,\,\, \exists! \alpha_1,
    \dots, \alpha_n \in K : \alpha_1 e_1 + \dots + \alpha_n e_n = v$,
    другими словами, любой вектор из пространства единственным образом
    расскладывается в линейную комбинацию базисных векторов.
\end{conj}

\begin{conj}
    Пусть $V, V'$ -- ЛП/$K$. Отображение $\phi : V \to V'$
    называется \textbf{изоморфизмом линейных пространств}, если
    выполняются два свойства:
    \begin{enumerate}
        \item $\phi$ -- биекция;
        \item $\forall v_1, v_2 \in V \,\,\, \forall \alpha_1, \alpha_2
        \in K \,\,\, \phi(\alpha_1 v_1 + \alpha_2 v_2) = \alpha_1
        \phi (v_1) + \alpha_2 \phi(v_2)$.
    \end{enumerate}
\end{conj}

$ $\\
\textbf{Пример:}\\
$\phi : M(m, n, K) \to M(n, m, K)$,\\
$\phi : A \mapsto A^T$, \\
$\phi$ -- изоморфизм ЛП.

\

\begin{theorem-non}
\end{theorem-non}
Пусть $V$ -- ЛП/$K$, $e_1, \dots, e_n \in V$.
Тогда следующие свойства эквивалентны:
\begin{enumerate}
    \item $e_1, \dots, e_n$ -- базис $V$;
    \item $\phi : K^n \to V$\\
    $\phi : \begin{pmatrix}
        \alpha_1 \\ \vdots \\ \alpha_n
    \end{pmatrix} \mapsto
    \sum \limits_{i=1}^n \alpha_i e_i$\\
    $\phi$ -- изоморфизм ЛП;
    \item $e_1, \dots, e_n$ -- линейно независимая порождающая система,\\
    т.е. $e_1, \dots, e_n$ -- ЛНС, и $V = \Lin(e_1, \dots, e_n)$;
    \item $e_1, \dots, e_n$ -- минимальная порождающая система для $V$,
    т.е. нельзя удалить из этого набора ни один вектор так, чтобы
    система осталась порождающей;
    \item $e_1, \dots, e_n$ -- максимальная линейно независимая система,
    т.е. нельзя добавить ни один вектор из $V$ так, чтобы система
    осталась линейно независимой.
\end{enumerate}
\begin{proof} $ $

    \begin{itemize}
        \item[$2 \Rightarrow 1$:] $\phi$ -- биекция $\Rightarrow$
        $\forall v \in V \,\,\, \exists! \alpha_1, \dots, \alpha_n \in K :
        \alpha_1 e_1 + \dots + \alpha_n e_n = v$, что по определению
        означает, что $e_1, \dots, e_n$ -- базис $V$.

        \item[$1 \Rightarrow 2$:] $\phi(\alpha a + \beta b) = 
        \alpha \phi(a) + \beta \phi(b)$ очевидно выполняется
        для любых $\alpha_1, \dots, \alpha_n \in K$. 
        
        Докажем теперь, что $\phi$ -- биекция. По опр. базиса $\forall
        v \in V \,\,\, \exists! \alpha_1, \dots, \alpha_n \in K : 
        \alpha_1 e_1 + \dots + \alpha_n e_n = v$. Заметим, что по сути
        $\alpha_1, \dots, \alpha_n$ -- это $\begin{pmatrix}
            \alpha_1 \\ \vdots \\ \alpha_n
        \end{pmatrix}$, а $\alpha_1 e_1 + \dots + \alpha_n e_n$ -- это
        $\phi\left(\begin{pmatrix}
            \alpha_1 \\ \vdots \\ \alpha_n
        \end{pmatrix}\right)$. Т.е. из опр. базиса видно, что каждому 
        вектору из $V$ сопоставлен ровно один столбец из $K^n$, значит
        $\phi$ -- биекция.

        \item[$1 \Rightarrow 3$:] Раз любой вектор раскладывается
        в линейную комбинацию базисных векторов, очевидно,
        что $e_1, \dots, e_n$ -- порождающая система. Проверим линейную
        независимость. Рассмотрим $\alpha_1 e_1 + \dots + \alpha_n e_n
        = \overline{0} = 0 \cdot e_1 + \dots + 0 \cdot e_n$. Т.к.
        $\overline{0}$ -- это тоже вектор, то в силу единственности
        разложения $\alpha_1 = \dots = \alpha_n = 0$. Таким образом,
        не существует нетривиальной ЛК $e_1, \dots, e_n$, равной
        $\overline{0}$, значит $e_1, \dots, e_n$ -- ЛНС.

        \item[$3 \Rightarrow 1$:] Т.к. это порождающая система,
        необходимо доказать только единственность. 
        Пусть $\alpha_1 e_1 + \dots + \alpha_n e_n =
        \beta_1 e_1 + \dots + \beta_n e_n$. Перенесём всё в левую
        часть и воспользуемся законом дистрибутивности.
        $(\alpha_1 - \beta_1) e_1 + \dots + (\alpha_n - \beta_n) e_n =
        \overline{0}$. $e_1, \dots, e_n$ -- ЛНС $\Rightarrow
        \alpha_1 - \beta_1 = \dots = \alpha_n - \beta_n = 0 \Rightarrow
        \alpha_1 = \beta_1, \dots, \alpha_n = \beta_n$.

        \item[$3 \Rightarrow 4$:] Предположим, что $e_1, \dots, e_{i-1},
        e_{i+1}, \dots, e_n$ -- порождающая система. НУО, $i = n$.
        Т.е. $V = \Lin(e_1, \dots, e_{n-1})$. Но тогда $e_n \in
        \Lin(e_1, \dots, e_{n-1}) \Rightarrow e_1, \dots, e_n$ -- ЛЗС.
        Противоречие.

        \item[$4 \Rightarrow 3$:] Предположим, что $e_1, \dots, e_n$ --
        ЛЗС. НУО, $e_n \in \Lin(e_1, \dots, e_{n-1})$. Тогда
        $V = \Lin(e_1, \dots, e_n) = \Lin(e_1, \dots, e_{n-1})
        \Rightarrow e_1, \dots, e_{n-1}$ -- порождающая система.
        Противоречие.

        \item[$3 \Rightarrow 5$:] Это порождающая система $\Rightarrow
        V = \Lin(e_1, \dots, e_n) \Rightarrow \forall v \in V \,\,\,
        v \in \Lin(e_1, \dots, e_n) \Rightarrow\\ e_1, \dots, e_n, v$ --
        ЛЗС.

        \item[$5 \Rightarrow 3$:] Возьмём $v \in V$. $e_1, \dots, e_n$ --
        ЛНС, но $e_1, \dots, e_n, v$ -- ЛЗС $\Rightarrow v \in
        \Lin(e_1, \dots, e_n) \Rightarrow e_1, \dots, e_n$ -- порождающая
        система.
    \end{itemize}
\end{proof}