\section{Разложение определителя по строке (столбцу)}

\begin{conj}
    $A_{ij} = (-1)^{i+j} \cdot |M_{ij}|$ называется алгебраическим
    дополнением к элементу в позиции $(i, j)$.
\end{conj}

\begin{theorem-non}
    Пусть $A = (a_{ij}) \in M(n, R)$

    Тогда $|A| = a_{k1}A_{k1} + \dots + a_{kn}A_{kn},\quad 1 \leqslant k \leqslant n$

    \begin{proof}
        $|A| =
        \begin{vmatrix}
            \vdots & \dots & \vdots \\
            a_{k1} & \dots & a_{kn} \\
            \vdots & \dots & \vdots
        \end{vmatrix} =$ 

        $\begin{vmatrix}
            a_{11} & a_{12} & \dots & a_{1n} \\
            \vdots & \vdots & \dots & \vdots \\
            a_{k1} & 0 & \dots & 0 \\
            \vdots & \vdots & \dots & \vdots \\
            a_{n1} & a_{n2} & \dots & a_{nn}
        \end{vmatrix} + $
        $\begin{vmatrix}
            a_{11} & a_{12} & \dots & a_{1n} \\
            \vdots & \vdots & \dots & \vdots \\
            0 & a_{k2} & \dots & 0 \\
            \vdots & \vdots & \dots & \vdots \\
            a_{n1} & a_{n2} & \dots & a_{nn}
        \end{vmatrix} + \dots + $
        $\begin{vmatrix}
            a_{11} & \dots & a_{1(n-1)} & a_{1n} \\
            \vdots & \dots & \vdots & \vdots \\
            0 & \dots & 0 & a_{nn} \\
            \vdots & \dots & \vdots & \vdots \\
            a_{n1} & \dots & a_{n(n-1)} & a_{nn}
        \end{vmatrix} = a_{k1}\cdot A_{k1} + \dots + a_{nk} \dots A_{nk}$
    \end{proof}
\end{theorem-non}

\notice Аналогично выглядит разложение определителя по $k$-му столбцу.

$|A| = a_{1k} \cdot A_{1k} + \dots + a_{nk} \cdot A_{nk}$

\textbf{TODO: Лемма О фальшивом разложении определителя}
