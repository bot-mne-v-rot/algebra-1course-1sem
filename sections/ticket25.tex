% !TEX root = ../LinalColloc02.tex

\section{Взаимная матрица. Явный вид обратной матрицы}

\begin{conj}
    
    Пусть $A \in M(n, R)$

    Взаимная (присоединённая, союзная) матрица $\widetilde{A} =
        \begin{pmatrix}
            A_{11} & A_{21} & \dots & A_{n1} \\
            A_{12} & A_{22} & \dots & A_{n2} \\
            \vdots & \vdots & \dots & \vdots \\
            A_{1n} & A_{2n} & \dots & A_{nn}
        \end{pmatrix}
    $
    
    $A_{ji}$ --- алгебраическое дополнение. Стоит на позиции $(i, j)$.
\end{conj}

\begin{theorem-non}
    
    $A \cdot \widetilde{A} = \widetilde{A} \cdot A = |A| \cdot E_n =
        \begin{pmatrix}
            |A| & \dots & 0 \\
            \vdots & \ddots & \vdots \\
            0 & \dots & |A|
        \end{pmatrix}
    $


    \begin{proof}
        $(A \cdot \widetilde{A})_{ij} = a_{i1}A_{j1} + a_{i2}A_{j2} + \dots + a_{in}A_{jn} =
            \begin{cases}
                |A|, \; (i = j) \\
                0, \; (i \neq j)
            \end{cases}
        $

        \textbf{Пояснение:}
        \emph{Мы здесь пользуемся разложением по строке,
        которое равно определителю,}\\
        \emph{и леммой о фальшивом разложении, при котором сумма равна 0.}
    \end{proof}

    \notice Для $\widetilde{A} \cdot A = |A| \cdot E_n$ --- аналогично,
    используя разложение по столбцу

\end{theorem-non}

\follow Пусть $A \in M(n, K), K$ --- поле, $|A| \neq 0$ (для кольца $|A| \in R^*$)

    Тогда $A^{-1} = |A|^{-1} \cdot \widetilde{A}$
    \begin{proof}
        $A \cdot |A|^{-1} \cdot \widetilde{A} = E_n$
        
        $|A|^{-1} \cdot \widetilde{A} \cdot A = E_n$
    \end{proof}

\follow $|A| \in R^* \Longrightarrow A \in GL (n, R)$ (Доказывается аналогично)

\notice Верно и обратное. $A \in GL(n, R) \Longrightarrow |A| \in R^*$
    \begin{proof}
        $A^{-1} \cdot A = A \cdot A^{-1} = E_n$ 

        $|A^{-1} \cdot A| \stackrel{*}{=} |A^{-1}| \cdot |A| = |A \cdot A^{-1}| \stackrel{*}{=} |A| \cdot |A^{-1}| = 1$

        $*$ --- проверили только для поля.

        $\Longrightarrow |A| \in R^*$ 
        
    \end{proof}
