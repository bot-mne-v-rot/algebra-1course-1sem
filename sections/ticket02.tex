% !TEX root = ../LinalColloc02.tex
\section{Комплексные корни вещественных многочленов}
\begin{theorem-non}
    Пусть $f \in \mathbb{R}[X]$ и $a \in \mathbb{C}$ - корень $f$. Тогда комплексно сопряженное число $\overline{a}$ -- корень $f$ той же кратности.
\end{theorem-non}

\begin{proof}
    Пусть $l$ -- кратность корня $a$. Таким образом, имеем $f = (X - a)^lg$ для некоторого $g \in \mathbb{C}[X]$ (причем $g(a) \neq 0$).
    Поскольку комплексное сопряжение -- автоморфизм поля комплексных чисел, нетрудно видеть, что для любых $g_1, \, g_2 \in C[X]$ выполнено $\overline{g_1g_2} = \overline{g_1} * \overline{g_2}$.
    В частности,
    \[ (X - \overline{a})^l\overline{g} = \overline{(X - a)^lg} = \overline{f} = f \]
    Таким образом, кратность корня $\overline{a}$ равна как минимум $l$. Покажем, что она равна в точности $l$:
    \[ \overline{g}(\overline{a}) = \overline{g(a)} \neq 0 \]
    Значит, $\overline{a}$ -- корень $f$ кратности $l$.

    \vspace{7mm}

    Таким образом, все мнимые корни многочлена $f \in \mathbb{R}[X]$ разбиваются на пары комплексно сопряженных друг с другом.
    Тогда каноническое разложение $f$ в кольце $\mathbb{C}[X]$ можно преобразовать к виду:
    \[ f = c \prod_{i = 1}^k ((X - a_i)^{m_i}(X - \overline{a_i})^{m_i}) \prod_{i = 1}^l (X - b_i)^{n_i} \]
    где $(a_i, \overline{a_i})$ -- пары мнимых корней, а $b_i$ -- вещественные корни.

    Упростим:
    \[ (X - a_i)(X - \overline{a_i}) = X^2 - (a_i + \overline{a_i})X + a_i\overline{a_i} = X^2 - 2Re \, a_i + |a_i|^2 \in \mathbb{R}[x] \]
    Тогда получаем следующее каноническое разложение $f$ в кольце $\mathbb{R}[X]$:
    \[ f = c \prod_{i = 1}^k (X^2 - 2Re \, a_i + |a_i|^2)^{m_i} \prod_{i = 1}^l (X - b_i)^{n_i} \] 
\end{proof}