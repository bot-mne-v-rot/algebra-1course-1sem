% !TEX root = ../LinalColloc02.tex
\section{Приведение матрицы к простейшему виду элем. преобразованиями строк и столбцов}
Для начала приведем матрицу к ступенчатому виду. 

Заметим, что ведущие элементы можно обнулить с помощью элементарных преобразований третьего типа.
А именно: $\forall i \in 1..r$ проведем преобразование $D_i(1 / a_{ij_i})$.

Теперь воспользуемся элементарными преобразованиями столбцов. 

С помощью преобразований первого типа обнулим в строках $1..r$ все элементы кроме ведущих.
Рассмотрим это на примере первой строки. 
Для всех $j > j_1$ мы проведем преобразование $T_{jj_1}(-a_{1j})$.
Таким образом, элемент $a_{1j}$ станет равен $a_{1j} + 1 * (-a_{1j}) = 0$. 
Проделав это для всех строк $1..r$, получим следующую матрицу:
\begin{gather*}
    \begin{pmatrix}
        0 & \dots & 0 & 1 & 0 & \dots & \dots & \dots & \dots & 0 \\
        0 & \dots & 0 & \dots & 0 & 1 & 0 & \dots & \dots & 0 \\
        \dots & \dots & \dots & \dots & \dots & \dots & \dots & \dots & \dots & 0 \\
        0 & \dots & 0 & \dots & 0 & \dots & 0 & 1 & \dots & 0 \\
        \dots & \dots & \dots & \dots & \dots & \dots & \dots & \dots & \dots & 0 \\
       0 & 0 & 0 & 0 & 0 & 0 & 0 & 0 & 0 & 0
    \end{pmatrix}
\end{gather*}
После этого перестановкой столбцов можно добиться того, что единицы будут стоять в позициях $(1, 1), (2, 2), \dots, (r, r)$.
Полученная матрица называется окаймленной единичной матрицей:
\begin{gather*}
    \begin{pmatrix}
        E_r & 0 \\
        0 & 0
    \end{pmatrix}
\end{gather*}