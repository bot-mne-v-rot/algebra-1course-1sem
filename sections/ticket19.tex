\section{Поведение определителя при элементарных преобразованиях матрицы}

\begin{theorem-non}
    $\quad$ 
    \begin{enumerate}
        \item При ЭП строк (столбцов) I типа определитель не изменяется.
        \item При ЭП строк (столбцов) II типа определитель меняет знак.
        \item При ЭП строк (столбцов) III типа определитель умножается на $\varepsilon$.
    \end{enumerate}

    \begin{proof} $\quad$ \\
        \begin{enumerate}
            \item[3.] Знаем (линейность)
            \item[1.]
            $A =
            \begin{pmatrix}
                A_1 \\
                \vdots \\
                A_n
            \end{pmatrix}, \quad A_i$ --- $i$-ая строка.
            
            $A' = \tau_{ij}(\lambda) \cdot A = 
            \begin{pmatrix}
                A_1 \\
                \vdots \\
                A_i + \lambda \cdot A_j \\
                \vdots \\
                A_n
            \end{pmatrix}$

            $|A'| = 
            \begin{vmatrix}
                A_1 \\
                \vdots \\
                A_i \\
                \vdots \\
                A_n
            \end{vmatrix} + 
            \begin{vmatrix}
                A_1 \\
                \vdots \\
                \lambda \cdot A_j \\
                \vdots \\
                A_n
            \end{vmatrix} = |A| + \lambda \cdot 
            \begin{vmatrix}
                A_1 \\
                \vdots \\
                A_j \\
                \vdots \\
                A_j \\
                \vdots \\
                A_n
            \end{vmatrix} = |A| + \lambda \cdot |0| = |A|$

            \item[2.]
            
            $A' =
            \begin{pmatrix}
                A_1 \\
                \vdots \\
                A_j$ (i-ая)$\\
                \vdots \\
                A_i$ (j-ая)$ \\
                \vdots \\
                A_n
            \end{pmatrix}$

            Пусть $B = 
            \begin{pmatrix}
                A_1 \\
                \vdots \\
                A_i + A_j$ (i-ая) $\\
                \vdots \\
                A_i + A_j$ (j-ая) $\\
                \vdots \\
                A_n
            \end{pmatrix}$

            $0 = |B| = 
            \begin{vmatrix}
                A_1 \\
                \vdots \\
                A_i \\
                \vdots \\
                A_i + A_j \\
                \vdots \\
                A_n
            \end{vmatrix} + 
            \begin{vmatrix}
                A_1 \\
                \vdots \\
                A_j \\
                \vdots \\
                A_i + A_j \\
                \vdots \\
                A_n
            \end{vmatrix} = 
            \begin{vmatrix}
                A_1 \\
                \vdots \\
                A_i \\
                \vdots \\
                A_i \\
                \vdots \\
                A_n
            \end{vmatrix} + 
            \begin{vmatrix}
                A_1 \\
                \vdots \\
                A_i \\
                \vdots \\
                A_j \\
                \vdots \\
                A_n
            \end{vmatrix} + 
            \begin{vmatrix}
                A_1 \\
                \vdots \\
                A_j \\
                \vdots \\
                A_i \\
                \vdots \\
                A_n
            \end{vmatrix} + 
            \begin{vmatrix}
                A_1 \\
                \vdots \\
                A_j \\
                \vdots \\
                A_j \\
                \vdots \\
                A_n
            \end{vmatrix} = 0 + |A| + |A'| + 0 \Longrightarrow$
            
            $|A'| = -|A|$ 
        \end{enumerate}
    \end{proof}

    \notice{Утверждение про столбцы получается через транспонирование.}









\end{theorem-non}