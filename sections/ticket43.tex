% !TEX root = ../LinalColloc02.tex

\section{Линейные отображения. Примеры. Ядро и образ}

\begin{conj}
    
    $V,W$ --- линейные пространства над полем $K$.

    Отображение $\mathcal{A} : V \rightarrow W$ называется
    линейным, если 
    \begin{gather*}
        \forall v_1, v_2 \in V,\; \forall \alpha_1, \alpha_2 \in K: \\
        \mathcal{A}(\alpha_1 v_1 + \alpha_2 v_2) =
        \alpha_1 \mathcal{A}(v_1) + \alpha_2 \mathcal{A}(v_2)    
    \end{gather*}
    

    \notice Вместо этого условия можно требовать
    \begin{gather*}
        \begin{cases}
            \mathcal{A}(v_1 + v_2) = \mathcal{A}(v_1) + \mathcal{A}(v_2) $\; (гомоморфизм групп)$ \\
            \mathcal{A}(\alpha v) = \alpha \mathcal{A}(v)
        \end{cases}    
    \end{gather*}
    
    $\Hom(V, W) := \{  \mathcal{A} \;|\; \mathcal{A}: V \rightarrow W \}$ ---
    множество линейных отображений из $V$ в $W$.

\end{conj}

\textbf{Примеры:}
\begin{enumerate}
    \item $V = W = K[x] \\
    \mathcal{A}: f \mapsto f'$
    \item $\operatorname{id}_V: V \mapsto V$ (изоморфизм на себя)
    \item $\lambda \in K$ \\
    $[\lambda] : V \rightarrow V \\
    v \mapsto \lambda v$ --- гомотетия; \\
    $\operatorname{id}_V = [1]$ --- тождественное отображение
    является частным случаем.
    \item  $0: V \rightarrow W \\
    v \mapsto 0$
    \item $A \in M(m, n, K)$ \\
    $\mathcal{A}: K^n \rightarrow K^m$ \\
    $C \mapsto AC$,\; где $C$ --- столбец
\end{enumerate}

\begin{theorem-non}
    $\mathcal{A} \in \Hom(U, V), \mathcal{B} \in \Hom(V, W)$

    Тогда $\mathcal{B} \circ \mathcal{A} \in \Hom(U,W)$

    \begin{proof}
        Очевидно
    \end{proof}

\end{theorem-non}

\begin{conj}
    Пусть $\mathcal{A} \in \Hom(V,W)$

    \begin{center}
        $\Imm \mathcal{A} = \{ \mathcal{A}(v) \;|\; v \in V \}$ --- образ \\
        $\Ker \mathcal{A} = \{ v \;|\; \mathcal{A}(v) = 0 \}$ --- ядро
    \end{center}

\end{conj}

\begin{theorem-non}
    $ $ 

    \begin{enumerate}
        \item $\Imm \mathcal{A}$ --- подпространство $W$
        \item $\Ker \mathcal{A}$ --- подпространство $V$
    \end{enumerate}
    
    \begin{proof}
        $ $

        \begin{enumerate}
            \item 
            \begin{flushleft}
                $w_1, w_2 \in \Imm \mathcal{A}$ \\
                $\alpha_1w_1 + \alpha_2w_2 \stackrel{?}{\in} \Imm \mathcal{A}$ \\
                $w_1 = \mathcal{A}(v_1),\; w_2 = \mathcal{A}(v_2)$ \\
                $\alpha_1w_1 + \alpha_2w_2 = \mathcal{A}(\alpha_1v_1 + \alpha_2v_2) \in \Imm \mathcal{A}$ 
            \end{flushleft}
            \item
            \begin{flushleft}
                $v_1, v_2 \in \Ker \mathcal{A}$ \\
                $\mathcal{A}(\alpha_1v_1 + \alpha_2v_2) = \alpha_1 \mathcal{A}(v_1) + \alpha_2 \mathcal{A}(v_2) = $ \\
                $ = \alpha_1 \cdot 0 + \alpha_2 \cdot 0 = 0$ \\
                $ \Longrightarrow \alpha_1 v_1 + \alpha_2 v_2 \in \Ker \mathcal{A}$
            \end{flushleft}
        \end{enumerate}

    \end{proof}

\end{theorem-non}

