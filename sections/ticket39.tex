\section{Минорный ранг}

\begin{conj} 
    Подматрица $B$ матрицы $A$ --- матрица, полученная из $A$ путем вычеркивания некоторых строк и некоторых столбцов.
\end{conj}
\
\begin{conj} 
    Минор матрицы $A$ порядка $r$ --- определитель какой-либо квадратной подматрицы $A$ порядка $r$.
\end{conj}
\
\begin{theorem-non} 
    
    Пусть $rk A = r$. Тогда в $A$ есть ненулевой минор порядка $r$ и нет ненулевого минора порядка больше $r$.

    \begin{proof}
        
        В $A$ есть $r$ ЛН столбцов. $A'$ --- подматрица из этих столбцов.

        $rk A' = r \Longrightarrow$ в $A'$ есть $r$ ЛН строк. $A''$ --- подматрица из этих строк.

        $A'' \in M(r, K)$

        $rk A'' = r \Longrightarrow |A''| \neq 0 \Longrightarrow$ существует ненулевой минор порядка $r$.

        Пусть $s > r$, $B$ --- подматрица $A$ $s \times s$, т.ч. $|B| \neq 0$

        Пусть $C$ --- подматрица $A$ $m \times s$, т.ч. $B$ --- подматрица $C$.

        Столбцы $B$ ЛН $\Longrightarrow$ столбцы $C$ ЛН (очевидно) $\Longrightarrow$ в $A$ есть $s$ ЛН столбцов $\Longrightarrow rk A \geq s > r$ --- противоречие.

    \end{proof}

\end{theorem-non}

