\section{Разложение перестановки в произведение транспозиций и элементарных транспозиций}
\begin{theorem-non}
    Любая перестановка раскладывается в произведние транспозиций

    \begin{proof}
        Возьмем какую-то перестановку $\pi \in S_n$
        Начнем с тождественной перестановки $id$ и покажем, 
        что последовательным домножением на транспозиции справа можно получить 
        перестановку $\pi$. Сначала добьемся того, чтобы на первом месте в нижней строке 
        табличной записи нашей перестановки стояло то, что нужно — то есть, $\pi(1)$. 
        Для этого нужно переставить местами первый столбик с тем, в котором стоит $\pi(1)$.
        После этого поставим на второе место в нижней строке $\pi(2)$ : так как $\pi$ является 
        перестановкой, то $\pi(1) \neq \pi(2)$, поэтому где-то справа от первого столбца есть 
        столбец с $\pi(2)$. Поменяем его со вторым. И так далее: на $k$-шаге мы добиваемся того, 
        что первые $k$ чисел в нижней строке нашей перестановки выглядели так: $\pi(1), \pi(2), \dots, \pi(k)$.
        В конце концов мы получим перестановку $\pi$ путем домножения $id$ на транспозиции, что и требовалось.
    \end{proof}
\end{theorem-non}

\begin{theorem-non}
    Любая транспозиция является произведением нечетного числа элементарных транспозиций

    \begin{proof}
        Неформально задача выглядит так: нам разрешено менять местами любые два соседних элемента в строке, а 
        хочется поменять местами два элемента, стоящих далеко друг от друга. Как этого добиться? Очень просто: сначала 
        ``продвинуть'' последовательно левый из этих элементов направо до второго, поменять их там местами, а потом 
        второй элемент «отогнать» обратно на место левого. При этом наши элементы поменяются местами, а все остальные 
        элементы останутся на своих местах: любой элемент между нашими мы затронем ровно два раза: на пути ``туда'' и на 
        пути ``обратно''; сначала он сдвинется на шаг влево, а потом — на шаг вправо. Ну, а любой элемент, стоящий не между 
        нашими, и подавно останется на своем месте. Аккуратный подсчет показывает, что мы совершили нечетное число операций.
        
        Формально же это рассуждение выражается в виде формулы:
        \begin{gather*}
            \tau_{ij} = \tau_{i,i+1} \circ \tau_{i+1,i+2} \circ \dots \circ \tau_{j-2,j-1} \circ \tau_{j-1,j} \circ 
            \tau_{j-2,j-1} \circ \dots \circ \tau_{i+1,i+2} \circ \tau_{i,i+1}
        \end{gather*}

        (здесь мы считаем, что $i < j$). Это равенство несложно проверить напрямую, и оно представляет транспозицию $\tau_{ij}$
        в виде произведения $2(j-i) - 1$ элементарных транспозиций.
    \end{proof}
\end{theorem-non}