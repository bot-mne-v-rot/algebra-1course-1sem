% !TEX root = ../LinalColloc02.tex
\section{Поле дробно-рациональных функций. Правильные дроби}
Про то, что такое поле дробно-рациональных функций, написано чуть выше.
\begin{theorem-non}
    (несократимое представление). Пусть $R$ -- факториальное кольцо. Тогда любой элемент $s \in Q(R)$ представим в виде $s = \frac{p}{q}$, 
    где $p$ и $q$ взаимно просты. Такое представление единственно с точностью до умножения $p$ и $q$ на элементы из $R^*$.
\end{theorem-non}

\begin{proof}
    Пусть $s = \frac{a}{b}, \; d = (a, b), \; a = da', \; b = db'$. 
    Тогда $s = \frac{a'}{b'}$ и $(a', b') = 1$. 
    Если $\frac{p}{q} = \frac{p'}{q'}$, то из $pq' = p'q$ следует $p|p'q$, откуда $p|p'$, и аналогично $p'|p$. 
    Таким образом, $p' = \varepsilon p$, где $\varepsilon \in R^*$, и отсюда $q' = \varepsilon q$.
\end{proof}

\textbf{Лемма.}
\textit{Пусть $s \in K(X), \, s = \frac{p}{q}$, где $p, \, q \in K[X]$. Тогда $\deg p - \deg q$ является инвариантом дроби $s$
(то есть не зависит от выбора представления в виде $p$ и $q$).}

\begin{proof}
    Если $\frac{p}{q} = \frac{p_1}{q_1}$, то $pq_1 = p_1q$, откуда $\deg p + \deg q_1 = \deg p_1 + \deg q$, следовательно, $\deg p - \deg q = \deg p_1 - \deg q_1$.
\end{proof}

\vspace{3mm}

Таким образом, можно говорить о степени рациональной дроби: $\deg s = \deg \frac{p}{q} = \deg p - \deg q$.
\begin{conj}
    Правильная дробь -- дробь, у которой степень числителя меньше степени знаменателя ($\deg s < 0$).
\end{conj}

В частности, 0 считается правильной дробью, а любой другой многочлен -- нет. 
Из определения легко вытекает, что сумма и произведение правильных дробей -- правильная дробь.

\textbf{Лемма.}
\textit{Любая рациональная дробь однозначно представляется в виде суммы многочлена и правильной дроби.}

\begin{proof}
    Пусть $s = \frac{p}{q}$. Разделим $p$ с остатком на $q$: $p = ql + r$. 
    Тогда \[\frac{p}{q} = l + \frac{r}{q} \] является искомым представлением.
    Проверим единственность: \[ l + \frac{r}{q} = l_1 + \frac{r_1}{q_1} \Rightarrow l - l_1 = \frac{r_1}{q_1} - \frac{r}{q} \]
    Заметим, что разность правильных дробей $\frac{r_1}{q_1}$ и $\frac{r}{q}$ тоже должна быть правильной дробью.
    Многочлен $l - l_1$ может быть правильной дробью, только когда он равен 0. Следовательно, $l = l_1$, а $\frac{r}{q} = \frac{r_1}{q_1}$. 
\end{proof}