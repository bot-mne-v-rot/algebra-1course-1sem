\section{Ранг набора векторов. Столбцовый и строчный ранг матрицы}

\begin{conj} 

    Ранг набора векторов $v_1,\, v_2,\, \dots,\, v_m \in V$: 
    
    $rk(v_1,\, \dots,\, v_m) := dim\, Lin(v_1,\, \dots,\, v_m)$ (есть обозначение $rank$, $rg$)

\end{conj}
\
\begin{theorem-non} 
    
    $rk(v_1,\, \dots,\, v_m)$ равен максимальному числу линейно независимых (ЛН) векторов среди $v_1,\, \dots,\, v_m$.
    \begin{proof}
        
        Пусть $r = rk(v_1,\, \dots,\, v_m)$. Нужно доказать: среди $v_1,\, \dots,\, v_m$

        \begin{enumerate}
            \item можно выбрать $r$ ЛН векторов:
            
            $W = Lin(v_1,\, \dots,\, v_m)$. 
            Среди $v_1,\, \dots,\, v_m$ можно выбрать базис (т.к. это порождающая система) -- это $r$ ЛН векторов.
            \item нельзя выбрать больше $r$ ЛН векторов:
            
            В $W$ нет ЛНС из более, чем $r$, векторов.
        \end{enumerate}
        
    \end{proof}
\end{theorem-non}

\begin{conj} 

    $A \in M(m, n, K)$

    Столбцовый ранг $A$ --- ранг совокупности её столбцов.

    Строчный ранг $A$ --- ранг совокупности её строк.

\end{conj}

\notice Столбцовый ранг $A$ $=$ строчному рангу $A^T$. Строчный ранг $A$ $=$ столбцовому рангу $A^T$.

