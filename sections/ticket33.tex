\section{Свойства матриц перехода между базисами}

\begin{conj} 
    
    Пусть $e_1,\, e_2,\, \dots\, ,\, e_n$ ($E$) --- базис $V$. 
    
    $v \in V$, тогда по определению базиса $v = \alpha_1 e_1 + \dots + \alpha_n e_n$. 
    Тогда $\alpha_1,\, \alpha_2,\, \dots\, ,\, \alpha_n$ --- координаты $v$ в базисе $E$.

    $[v]_E =
        \begin{pmatrix}
            \alpha_1\\
            \alpha_2\\
            \vdots\\
            \alpha_n
        \end{pmatrix}
    \in K^n$
\end{conj}
\
\begin{conj}


    $E' = (e'_1,\, e'_2,\, \dots\, ,\, e'_n)$, $[v]_{E'}$ --- другой базис, другие координаты вектора $v$ в нём.

    $e'_j = c_{1j}e_1 + \dots + c_{nj}e_n,\ j = 1, \dots, n$

    $C =
        \begin{pmatrix}
            c_{11} & c_{21} & \dots & c_{n1} \\
            c_{12} & c_{22} & \dots & c_{n2} \\
            \vdots & \vdots & \dots & \vdots \\
            c_{1n} & c_{2n} & \dots & c_{nn}
        \end{pmatrix}
    $

    $[e'_j]_E$ --- $j$-ый столбец матрицы $C$.

    Заметим, что $E' = E \cdot C$.


    \textbf{Пояснение:}
    \emph{Вообще мы говорили только об умножении матриц над одним кольцом, 
    но чисто формально можно забить сейчас, так как умеем умножать скаляр на вектор.}

    Тогда $C$ --- матрица перехода от базиса $E$ к $E'$. Второе обозначение --- $M_{E \to E'}$.


\end{conj}
\
\begin{theorem-non} 
    
    Пусть $C_1 = M_{E \to E'}, \, C_2 = M_{E' \to E''} \Longrightarrow M_{E \to E''} = C_1 \cdot C_2$
    \begin{proof}
        
        $E' = E \cdot C_1, \  E'' = E' \cdot C_2$ 

        $E'' = (E \cdot C_1) \cdot C_2 = E \cdot (C_1 \cdot C_2), \  E'' = E \cdot M_{E \to E''}$

        $E$ --- базис $\Longrightarrow M_{E \to E''} = C_1 \cdot C_2$.
    \end{proof}
\end{theorem-non}

\follow $M_{E' \to E} = M_{E \to E'}^{-1}$; в частности, $M_{E \to E'} \in GL(n, k)$.

\begin{proof}
        
    $M_{E \to E'} \cdot M_{E' \to E} = M_{E \to E} = E_n$

    $M_{E' \to E} \cdot M_{E \to E'} = M_{E' \to E'} = E_n$
\end{proof}

