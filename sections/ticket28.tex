% !TEX root = ../LinalColloc02.tex

\section{Линейно зависимые семейства, свойства}

\begin{theorem-non}
\end{theorem-non}
Пусть $V$ -- ЛП/$K$. $v_1, \dots, v_n \in V$. Эквивалентны
следующие два свойства:
\begin{enumerate}
    \item $\exists \alpha_1, \dots, \alpha_n \in K$, т.ч. 
    $\exists i : \alpha_i \neq 0$ и $\alpha_1 v_1 + \dots + \alpha_n v_n
    = 0$. \\ Другими словами, сущ. нетривиальная линейная 
    комбинация данных векторов, равная $0$.
    
    \item $\exists j \in \{1..n\} : v_j \in \Lin(v_i \mid i \neq j)$. \\
    Другими словами, $v_j$ является линейной комбинацией остальных
    векторов.
\end{enumerate}

\begin{proof} $ $

    ``$1 \Rightarrow 2$'':
    \begin{align*}
        & \exists j : \alpha_j \neq 0; \\
        & \alpha_1 v_1 + ... + \alpha_n v_n = 0, \\
        & \alpha_1 v_1 + ... + \alpha_{j-1} v_{j-1} 
        + \alpha_{j+1} v_{j+1}
        + ... + \alpha_n v_n = - \alpha_j v_j; \\
        & \text{Т.к. } \alpha_j \neq 0 \text{, мы можем поделить на }
        \alpha_j; \\
        & v_j = \left(-\frac{\alpha_1}{\alpha_j}\right) v_1 + \dots +
        \left(-\frac{\alpha_{j-1}}{\alpha_j}\right) v_{j-1} + 
        \left(-\frac{\alpha_{j+1}}{\alpha_j}\right) v_{j+1} + \dots +
        \left(-\frac{\alpha_n}{\alpha_j}\right) v_n 
        \in \Lin(v_1, \dots, v_n)
    \end{align*}
    ``$2 \Rightarrow 1$'':
    \begin{align*}
        & v_j \in \Lin(v_i \mid i \neq j) \Rightarrow
        v_j = \sum_{i \neq j} \beta_i v_i; &&&&&&&&&&&&&&&&&&&&&&&& \\
        & \sum_{i \neq j} \beta_i v_i - v_j = 
        \sum_{i=1}^n \beta_i v_i = 0 \text{, где } \beta_j = -1.
    \end{align*}
    $\quad \,\,$ Получаем нетривиальную линейную комбинацию, т.к. 
    при векторе $v_j$ коэффициент $1 \neq 0$.

\end{proof}

\begin{conj} $ $\\
    $v_1, \dots, v_n$ -- \textbf{линейно зависимая система} 
    (\textbf{ЛЗС}), если выполняются условия из предложения.\\
    $v_1, \dots, v_n$ -- \textbf{линейно независимая система}
    (\textbf{ЛНС}) в противном случае.
\end{conj}

\begin{theorem-non}
\end{theorem-non}
\begin{enumerate}
    \item $v_1, \dots, v_n$ -- ЛЗС $\Rightarrow \forall \sigma \in S_n 
    \,\,\, v_{\sigma(1)}, \dots, v_{\sigma(n)}$ -- ЛЗС.
    \item С ЛЗС:\\
    $v_1, \dots, v_n$ -- ЛЗС, $v \in V$ $\Rightarrow
    v_1, \dots, v_n, v$ -- ЛЗС.

    С ЛНС:\\
    $v_1, \dots, v_n$ -- ЛНС $\Rightarrow \forall i = 1..n
    \,\,\, v_1, \dots, v_{i-1}, v_{i+1}, \dots, v_n$ -- ЛНС.
    
    \item $\exists i : v_i = 0 \Rightarrow v_1, \dots, v_n$ -- ЛЗС.
    \item $v_1, \dots, v_n$ -- ЛНС, $v \in V$ Тогда: 
    $v_1, \dots, v_n, v$ -- ЛЗС $\Longleftrightarrow
    v \in \Lin(v_1, \dots, v_n)$.
\end{enumerate}
\begin{proof} $ $

    \begin{enumerate} 
        \item Тривиально.
        \item С ЛЗС: \\
        $\exists \alpha_1, \dots, \alpha_n \in K \mid 
        \exists \alpha_i \neq 0$ и 
        $\alpha_1 v_1 + \dots + \alpha_n v_n = 0$ -- нетрив. ЛК 
        $\Longrightarrow \alpha_1 v_1 + \dots + \alpha_n v_n + 0 \cdot v 
        = 0$ -- тоже нетрив. ЛК.

        С ЛНС: \\
        Аналогично.

        \item $1 \cdot v_i + \sum \limits_{j \neq i} 0 \cdot v_j = 0$ --
        нетривиальная ЛК.

        \item 
        ``$\Longleftarrow$'':\\
        По определению ЛЗС.

        ``$\Longrightarrow$'':\\
        $v_1, \dots, v_n, v$ -- ЛЗС $\Rightarrow \exists
        \alpha_1, \dots, \alpha_{n+1} \in K \mid \exists \alpha_i \neq 0$
        и $\alpha_1 v_1 + \dots + \alpha_n v_n + \alpha_{n+1} v = 0$.
        Предположим, что $\alpha_{n+1} = 0$ $\Rightarrow 
        \alpha_1 v_1 + \dots + \alpha_n v_n = 0$ -- нетрив. ЛК
        $\Rightarrow v_1, \dots, v_n$ -- ЛЗС. Противоречие.
        Таким образом, $\alpha_{n+1} \neq 0$, значит
        $v = \sum \limits_{i = 1}^n \frac{-\alpha_i}{\alpha_{n + 1}}
        v_i \Rightarrow v \in \Lin(v_1, \dots, v_n)$.


    \end{enumerate}
\end{proof}
