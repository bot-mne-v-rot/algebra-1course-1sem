% !TEX root = ../LinalColloc02.tex

\section{Размерность. Свойства пространств заданной размерности}

\begin{theorem-non}
\end{theorem-non}
Пусть $(e_1, \dots, e_n)$, $(e_1', \dots, e_m')$ -- базисы $V$ -- ЛП/$K$.
Тогда $n = m$.
\begin{proof}
    Предположим, что это не так. Пусть $n > m$. $e_1', \dots, e_m'$ --
    базис $\Rightarrow e_1, \dots, e_n \in \Lin(e_1', \dots, e_m')$.
    Но $n > m \Rightarrow$ по теореме о линейной зависимости 
    линейных комбинаций $e_1, \dots, e_n$ -- ЛЗС, но это противоречит
    3-ему свойству теоремы о равносильных определениях базиса. 
    Значит, $n = m$.
\end{proof}

\begin{theorem-non}
\end{theorem-non}
Из любой системы образующих $V$ -- ЛП/$K$ -- можно выделить
базис.
\begin{proof}
    Пусть $V = \Lin(v_1, \dots, v_n)$. Выберем в $v_1, \dots, v_n$
    наименьшую по мощности подсистему, являющуюся системой образующих.
    Другими словами, рассмотрим все подмножества $v_1, \dots, v_n$,
    оставим из них только те, что являются системой образующих, и
    выберем из них подмножество с наименьшим количеством элементов.
    Это можно сделать, т.к. у конечного множества существует конечное 
    число подмножеств. Получаем минимальную порождающую систему.
    Значит, по свойству 4 из теоремы о равносильных определениях базиса
    это базис.
\end{proof}

\follow $ $ У любого конечномерного пространства есть базис.

\begin{conj}
Пусть $V$ -- конечномерное пространство. Его \textbf{размерностью} 
называется число векторов в любом его базисе. Обозначается она $\dim V$.
\end{conj}

\textbf{Примеры:}
\begin{enumerate}
    \item $V := M(m, n, K)$. Любая матрица $A = (a_{ij}) \in V$
    преставима в виде линейной комбинации матричных единиц $A =
    \sum_{i, j} a_{ij} \cdot e_{ij}$. Откуда видно, что
    $e_{11}, \dots, e_{mn}$ -- базис $V$, и $\dim V = mn$.

    \item $\dim K^n = n$.
    
    \item Считается, что $\dim \, \{ \, \overline{0} \, \} = 0$.
\end{enumerate}

\begin{lemma}
\end{lemma}
Пусть $V$ -- ЛП, $\dim V = n$; $v_1, \dots, v_N \in V$, $N > n$.
Тогда $v_1, \dots, v_N$ -- ЛЗС.
\begin{proof} $ $

    Непосредственно из теоремы о линейной зависимости линейных комбинаций.
\end{proof}

\begin{theorem-non}
\end{theorem-non}
Пусть $V$ -- конечномерное ЛП; $v_1, \dots, v_n$ -- ЛНС. Тогда
её можно дополнить до базиса.
\begin{proof}
    Если $V = \Lin(v_1, \dots, v_n)$, то $v_1, \dots, v_n$ -- базис
    по свойству 3 т. о равносильных опр. базиса, иначе
    $\exists v_{n+1} \in V, v_{n+1} \notin \Lin(v_1, \dots, v_n)
    \Rightarrow v_1, \dots, v_{n+1}$ -- тоже ЛНС.
    Если $V = \Lin(v_1, \dots, v_{n+1})$, то $v_1, \dots, v_{n+1}$
    -- базис, иначе повторяем действия.
    
    Пусть $m = \dim V$. Тогда $v_1, \dots, v_{m + 1}$ -- ЛЗС
    $\Rightarrow$ алгоритм завершится.
\end{proof}

\follow 

Пусть $\dim V = n$; $e_1, \dots, e_n \in V$. Тогда
следующие утверждения эквивалентны:
\begin{enumerate}
    \item $e_1, \dots, e_n$ -- базис.
    \item $e_1, \dots, e_n$ -- ЛНС.
    \item $e_1, \dots, e_n$ -- порождающая система.
\end{enumerate}
\begin{proof} $ $

    $1 \Rightarrow 2, 3$: Очевидно.

    $2 \Rightarrow 1$: ЛНС можно дополнить до базиса, 
    но базис состоит из $n$ векторов $\Rightarrow$ 
    $e_1, \dots, e_n$ -- базис.

    $3 \Rightarrow 1$: из порождающей системы можно выделить базис, 
    но базис состоит из $n$ векторов $\Rightarrow$ 
    $e_1, \dots, e_n$ -- базис.
\end{proof}