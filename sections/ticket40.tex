\section{Системы линейных уравнений. Классификация. Метод Гаусса}

\begin{conj}
    
    Система $m$ линейных уравнений с $n$ неизвестными имеет вид:

    $(*)\begin{cases}
        a_{11}x_1 + \dots + a_{1n}x_n = b_1 \\
        a_{21}x_1 + \dots + a_{2n}x_n = b_2 \\
        \vdots + \dots + \vdots \\
        a_{m1}x_1 + \dots + a_{mn}x_n = b_m
    \end{cases}$

    где $a_{ij}, b_i \in K \quad (i = 1, \dots, m; \; j = 1, \dots, n)$

    $A = (a_{ij})$ --- матрица СЛУ $(*)$

    $b =
    \begin{pmatrix}
        b_1 \\
        \vdots \\
        b_m    
    \end{pmatrix}$ --- столбец правых частей СЛУ $(*)$

    $(A | b) \in M(m, n + 1, K)$ --- расширеная матрица СЛУ $(*)$

\end{conj}

\begin{conj}
    $X = 
    \begin{pmatrix}
        \xi_1 \\
        \vdots \\
        \xi_n \\
    \end{pmatrix}$ --- является решением $(*)$,
    если при подстановке $x_1:=\xi_1, \dots, x_n:=\xi_n$, 
    мы получаем $m$ верных равенств.

    То есть если $AX = b$
\end{conj}

\begin{theorem-non}
    $(A|b) \rightarrow (A'|b')$ с помощью ЭП строк
    $\Longrightarrow$ системы с расширенными
    матрицами $(A|b)$ и $(A'|b')$ эквиваленты.

    \begin{proof}
        \begin{gather*}
            (A'|b') = U \cdot (A | b), \quad U \in GL(m, K) \\
            AX = b \Longleftrightarrow UAX = Ub \longleftrightarrow A'X = b'      
        \end{gather*}
    \end{proof}

\end{theorem-non}

\subsection*{Классификация СЛУ}

\begin{conj}
    Если $b = 0$, то СЛУ $(*)$ однородная, иначе неоднородная.
\end{conj}

\begin{conj}
    СЛУ $(*)$ называется совместной, если $\{ X | AX = b \} \neq \varnothing$,
    иначе --- система несовместная.
\end{conj}

\begin{conj}
    Совместная система называется определённой, если её решение единственно,
    иначе --- система неопределённая.
\end{conj}

\notice Однородная система --- совместная. $(A \cdot 0 = 0)$

\subsection*{Метод Гаусса}
\textbf{Этапы:}
\begin{enumerate}
    \item[I:] $(A | b) \stackrel{\text{ЭП строк}}{\longrightarrow} \underbrace{(A' | b')}_{\text{ступенчатая}}$
    \item[II:] $\quad$ \\ 
    $\begin{pmatrix}
        0 & \dots & 0 & a_{1j} & \dots & | & \dots \\
        0 & \dots & 0 & a_{2j} & \dots & | & \dots \\
        \vdots & \dots & \vdots & \vdots & \dots & | & \dots \\
        0 & \dots & 0 & \dots & | & \dots $r-ая$\\ 
        0 & \dots & 0 & 0 & \dots & | & 0 \\
        \vdots & \dots & \vdots & \vdots & \dots & | & \dots \\
        0 & \dots & 0 & 0 & \dots & | & 0
        
        
    \end{pmatrix}$

    $a_{rj_r}$ --- ведущий элемент последней ненулевой строки
    \begin{enumerate}
        \item $j_r = n + 1$ \\
        $0 \cdot x_1 + \dots + 0 \cdot x_n = a_{rj_r} \neq 0$
        $\Longrightarrow (*)$ несовместная.
        \item $j_r \leqslant n$ \\
        $x_{j1}, \dots, x_{jr}$ --- главные неизвестные. \\
        Остальные --- свободные неизвестные. \\
        $x_s:= \xi_s \in K, \quad s \not \in \{ j_1, \dots, j_r \}$ \\
        $x_{j_r} = a^{-1}_{rj_r}(b_r - a_{rj_r + 1}x_{r+1} - \dots - a_{rn}x_n)$ \\
        $x_{j_{r-1}} = a^{-1}_{(r-1)j_{r-1}}(b_{r-1} - a_{(r-1)j_{r-1} + 1}x_{j_{r-1} + 1} - \dots - a_{r-1}x_{?}$ \\
        $\dots$ \\
        $x_1 = \dots$ \\
        Решение уравнения зависит от $n - r$ параметров. \\
    \end{enumerate}

    Получили
    $\begin{pmatrix}
        x_1 \\
        \vdots \\
        x_n
    \end{pmatrix} = F(\xi_1, \dots, \xi_{n-r})$ 

    В частности: если $(*)$ совместная, то она определённая $\Longleftrightarrow rk(A|b) = n$
\end{enumerate}

\textbf{блять что вообще происходит..., кто просил его писать на графическом планшете?}