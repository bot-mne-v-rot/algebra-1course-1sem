\section{Связь между размерностями ядра и образа}

\begin{theorem-non}

    Пусть $dim \; V = n < \inf$ \\
    $\mathcal{A} \in Hom(V, W)$ \\
    Тогда $Im \; \mathcal{A}$ конечномерен и
    $dim Ker \; \mathcal{A} + dim Im \; \mathcal{A} = n$

    \begin{proof}
        
        $Ker \; \mathcal{A} \subset V$

        $\Longrightarrow Ker \; \mathcal{A}$ конечномерен. 
        
        $\underbrace{e_1, \dots, e_m}_{\text{ЛНС в V}}$ --- любой базис $Ker \; \mathcal{A}$

        Пусть $e_{m + 1}, \dots, e_n$ --- дополнение до базиса $V$.

        Докажем, что $\mathcal{A}(e_{m+1}), \dots, \mathcal{A}(e_n)$ ---
        базис $Im \; \mathcal{A}$.

        \begin{proof}
            \begin{gather*}
                w \in Im \; \mathcal{A} \Longrightarrow w = \mathcal{A}(v), v \in V \\    
                v = \alpha_1e_1 + \dots + \alpha_ne_n \\
                \mathcal{A}(v) =
                \underbrace{\alpha_1\mathcal{A}(e_1)}_{\text{ = 0 }} +
                \dots +
                \underbrace{\alpha_n\mathcal{A}(e_n)}_{\text{ = 0}} +
                \alpha_{m+1}\mathcal{A}(e_{m+1}) + \dots + \alpha_n\mathcal{A}(e_n) = \\
                = \alpha_{m+1}\mathcal{A}(e_{m+1}) + \dots + \alpha_n\mathcal{A}(e_n) \in Lin(\mathcal{A}(e_{m+1}) + \dots + \mathcal{A}(e_n)) \\ 
            \end{gather*}

            Проверим их линейную независимость:

            \begin{center}
                $\beta_1\mathcal{A}(e_{m+1}) + \dots + \beta_{n - m}\mathcal{A}(e_{n}) = 0$ \\
                 $\mathcal{A}(
                     \underbrace{\beta_1e_{m+1} + \dots + \beta_{n-m}e_n}_{\text{$\Rightarrow \in Ker \; \mathcal{A}$}}
                     ) = 0$ \\
                    $\beta_1e_{m+1} + \dots + \beta_{n-m}e_n = \alpha_1e_1 + \dots + \alpha_me_m$ \\
                    $e_1, \dots, e_n$ --- ЛНС $\Longrightarrow $ все $\alpha_i$ и $ \beta_i = 0$ \\
                    Т.о. $\mathcal{A}(e_{m+1}), \dots, \mathcal{A}(e_n)$ --- базис $Im \; \mathcal{A}$ \\
                    $dim \; Im \; \mathcal{A} = n - m = dim \; V - dim \; Ker \; \mathcal{A}$
            \end{center}


        \end{proof}
    \end{proof}

\end{theorem-non}