\section{Теорема Крамера}

\begin{theorem}
    Крамера.

    Пусть $A \in M(n, K), \; b \in K^n$

    Два утверждения эквиваленты:
    \begin{enumerate}
        \item СЛУ с расширенной матрицей $(A | b)$ совместная определённая
        \item $|A| \neq 0$
    \end{enumerate}

    \begin{proof}
        $ $ 

        $2 \Longrightarrow AX = b \Longleftrightarrow A^{-1}AX = A^{-1}b
        \Longleftrightarrow X = A^{-1}b$

        $1 \Longrightarrow 2$
        
        От противного: $|A| = 0 \Longrightarrow rkA < n$

        Но у совместной определённой системы $rkA = n$ (противоречие)

    \end{proof}

\end{theorem}
\subsection*{Формулы Крамера.}
\begin{flushleft}
    $A^{-1} = \frac{1}{|A|} \cdot \widetilde{A}$

    $\widetilde{A} = (A_{ji}), \quad \widetilde{A}[ i, j ] = A_{ji}$

    $x_i = b_1 A^{-1} [ i, 1 ] + \dots + b_n A^{-1} [ i, n ] = $

    $ = \frac{1}{|A|} \underbrace{(b_1A_{1i} + \dots + b_nA_{ni})}_{\text{C}}$

    $C = \begin{vmatrix}
               &       & $i столб$ & & \\ 
        a_{11} & \dots & b_1 & \dots & a_{1n} \\
        \vdots & \dots & \vdots & \dots & \vdots \\
        a_{n1} & \dots & b_n & \dots & a_nn
    \end{vmatrix} = \frac{\Delta_i}{|A|}$

    $\Delta_i = |A \{ A_i \rightsquigarrow b \} |$ (определитель матрицы $A$, у которой мы заменили $i$-ый столбец на столбец $b$)

    $x_i = \frac{\Delta_i}{|A|}$ --- формулы Крамера.
\end{flushleft}

\begin{theorem-non}
    Множество решений однородной системы  с $n$ неизвестными --- линейное подпространство в $K^n$.

    \begin{proof}
        Пусть $x_1, x_2$ --- решения, $\; \alpha_1, \alpha_2 \in K$    
        
        Тогда $А(\alpha_1 x_1 + \alpha_2 x_2) = \alpha_1 \underbrace{Ax_1}_{\text{ = 0}} + \alpha_2 \underbrace{Ax_2}_{\text{ = 0}} = 0$
    \end{proof}
    
\end{theorem-non}