% !TEX root = ../LinalColloc02.tex
\section{Элементарные преобразования и элементарные матрицы}
Элементарные преобразования строк:
\begin{enumerate}
    \item Прибавление к строке $i$ строки $j$, домноженной на скаляр $\lambda \in R$.
    \item Перестановка местами $i$-той и $j$-той строки.
    \item Домножение строки $i$ на обратимый скаляр $\varepsilon \in R^*$.
\end{enumerate}

\begin{notice}
    Аналогичные преобразования определены для столбцов.
\end{notice}

Матрицы элементарных преобразований:
\begin{enumerate}
    \item Пусть $A \in M(m, n, R), \, \lambda \in R, \, T_{ij}(\lambda) = E_n + \lambda e_{ij} \; (i \neq j)$. 
    Тогда элементарному преобразованию первого типа соответствует умножение матрицы $A$ слева на $T_{ij}(\lambda)$. 

    \begin{proof}
        Поскольку матрица $T_{ij}(\lambda)$ отличается от единичной только в $i$-той строке, произведение $T_{ij}(\lambda)A$ отличается от матрицы $A$ также только в $i$-той строке.
        В $i$-той строке матрицы $T_{ij}(\lambda)$ только два элемента отличны от 0:
        элемент в позиции $i$ равен 1, а элемент  в позиции $j$ равен $\lambda$. 
        При умножении на $k$-тый столбец матрицы $A$ получаем следующее:
        \begin{gather*}
            \begin{pmatrix}
                0 & \dots & 1 & \dots & \lambda & \dots & 0
            \end{pmatrix} \cdot
            \begin{pmatrix}
                a_{1k} \\ \vdots \\ a_{ik} \\ \vdots \\ a_{jk} \\ \vdots \\ a_{nk}
            \end{pmatrix} = a_{ik} + \lambda a_{jk}
        \end{gather*}
        Это происходит в каждом столбце матрицы $A$, поэтому $i$-тая строка матрицы $T_{ij}(\lambda)A$ равна $(a_{i1} + \lambda a_{j1}) + \dots + (a_{in} + \lambda a_{jn})$.
    \end{proof}
    \item Пусть $A \in M(m, n, R), \, S_{ij} = E_n - e_{ii} - e_{jj} + e_{ij} + e_{ji}$.
    Тогда элементарному преобразованию второго типа соответствует умножение матрицы $A$ слева на $S_{ij}$.
    
    \begin{proof}
        Поскольку матрица $S_{ij}$ отличается от единичной только в $i$-той и $j$-той строке, произведение $S_{ij}A$ отличается от матрицы $A$ также только в $i$-той и $j$-той строке.
        $i$-тая строка равна произведению $(0 \dots 1 \dots 0)$ (где 1 стоит на $j$-м месте) на матрицу $A$, то есть $j$-той строке матрицы $A$.
        Аналогично для $j$-той строки.
    \end{proof}
    \item Пусть $A \in M(m, n, R), \, D_i(\varepsilon) = E_n + (\varepsilon - 1)e_{ii}$.
    Тогда элементарному преобразованию третьего типа соответствует умножение матрицы $A$ слева на $D_i(\varepsilon)$.
    
    \begin{proof}
        Поскольку матрица $D_i(\varepsilon)$ отличается от единичной только в $i$-той строке, произведение $D_i(\varepsilon)A$ отличается от матрицы $A$ также только в $i$-той строке.
        $i$-тая строка равна произведению $(0 \dots \varepsilon \dots 0)$ на $A$, что равно произведению $\varepsilon$ и $i$-той строки $A$.    
    \end{proof}
\end{enumerate}

\begin{notice}
    Применяя транспонирование (с учетом свойства $(AB)^T = B^TA^T$), получаем, что элементарным преобразованиям столбцов 
    соответствуют домножения справа на эти же матрицы.
\end{notice}