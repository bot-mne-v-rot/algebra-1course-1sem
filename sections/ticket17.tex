\section{Линейность определителя по строкам и столбцам}
\begin{theorem-non}
    \begin{itemize} \quad 

        \item[] Линейность определителя по строке
         
        \begin{enumerate}
            \item Пусть матрица $A = \begin{pmatrix}
                a_{11} & \dots & a_{1n} \\
                \dots & \dots & \dots \\
                a'_{k1} + a''_{k1} & \dots & a'_{kn} + a''_{kn} \\
                \dots & \dots & \dots \\
                a_{n1} & \dots & a_{nn}
            \end{pmatrix} \Longrightarrow \abs{A} + \abs{A'} + \abs{A''}$ 

            $A' = \begin{pmatrix}
                a_{11} & \dots & a_{1n} \\
                \dots & \dots & \dots \\
                a'_{k1} & \dots & a'_{kn} \\
                \dots & \dots & \dots \\
                a_{n1} & \dots & a_{nn}
            \end{pmatrix}$ \qquad 
            $A'' = \begin{pmatrix}
                a_{11} & \dots & a_{1n} \\
                \dots & \dots & \dots \\
                a''_{k1} & \dots & a''_{kn} \\
                \dots & \dots & \dots \\
                a_{n1} & \dots & a_{nn}
            \end{pmatrix}$
            \begin{proof}
                $\abs{A} = \sum\limits_{\sigma} sgn \ \sigma \cdot a_{1, \sigma(1)} \dots 
                \times (a'_{k, \sigma(k)} + a''_{k, \sigma(k)}) \times \dots a_{n, \sigma(n)} = \abs{A'} + \abs{A''}$
            \end{proof}
            \item $A = (a_{ij}), \qquad \lambda \in R, 1 \leqslant k \leqslant n$
            
            $B = \begin{pmatrix}
                a_{11} & \dots & a_{1n} \\
                \dots & \dots & \dots \\
                \lambda a_{k1} & \dots & \lambda a_{kn} \\
                \dots & \dots & \dots \\
                a_{n1} & \dots & a_{nn}
            \end{pmatrix} \Longrightarrow \abs{B} = \lambda \abs{A}$ 
            \begin{proof}
                $\abs{B} = \sum\limits_{\sigma} sgn \ \sigma \cdot a_{1, \sigma(1)} \dots 
                \lambda a_{k, \sigma(k)} \dots a_{n, \sigma(n)} = \lambda \abs{A}$, так как $\lambda$ можно сразу вынести
            \end{proof}
        \end{enumerate}
        \item[] Линейность определителя по столбцу 
        \begin{proof}
            Это утверждение сводится к утверждению про строки путем транспонирования

            Транспонируем матрицу, определитель не меняется, затем применяем одно из двух свойств из 
            предложения о линейности определителя по строке и потом снова транспонируем.
        \end{proof} 
    \end{itemize}
\end{theorem-non}

\follow \; Определитель с нулевой строкой или нулевым столбцом равен нулю. Это просто вытекает из второго свойства в предложении 
о линейности определителя по строке.