\section{Поле частных области целостности}
\begin{conj}
    Поле частных области целостности $R$ -- наименьшее поле, содержащее $R$.
\end{conj}
Элементы поля частных представляются как дроби $\frac{a}{b}$, где $a, \, b \in R$ и $b \neq 0$.
Это поле строится следующим образом.

На множестве $R \times (R \setminus \{0\})$ введем отношение $\sim$. Положим $(a, b) \sim (a', b')$, если $ab' = a'b$. 
Легко видеть, что это отношение эквивалентности.
Рефлексивность и симметричность очевидны. Проверим транзитивность.
Если $(a, b) \sim (a', b')$ и $(a', b') \sim (a'', b'')$, то $ab' = a'b$ и $a'b'' = a''b'$. 
Домножив первое равенство на $b''$, а второе на $b$, получаем:
\[ ab'b'' = a'bb'' = a''bb' \]
Поскольку $R$ -- область целостности, и $b' \neq 0$, мы можем сократить на $b'$ и получить, что $ab'' = a''b$, что означает $(a, b) \sim (a'', b'')$.

Обозначим через $Q(R)$ соответствующее фактормножество $(R \times (R \setminus \{0\})) / \sim$.
При этом класс пары $(a, b)$ мы будем записвать в виде дроби $\frac{a}{b}$, где горизонтальная черта еще пока не означает деление.
Условие $(a, b) \sim (a', b')$ будет означать, что $\frac{a}{b} = \frac{a'}{b'}$.

Введем сложение и умножение дробей:
\[ \frac{a_1}{b_1} + \frac{a_2}{b_2} = \frac{a_1b_2 + a_2b_1}{b_1b_2} \quad\quad 
\frac{a_1}{b_1} \cdot \frac{a_2}{b_2} = \frac{a_1a_2}{b_1b_2} \]

\begin{theorem-non}
    $(Q(R), +, \cdot)$ -- поле
\end{theorem-non}

\begin{proof}
    Для начала проверим, что результат сложения (умножения) не меняется при замене любой из пар $(a_1, b_1)$ и $(a_2, b_2)$ на эквивалентную.
    У нас есть такой переход:
    \[ \frac{a}{b} = \frac{ab'}{bb'} = \frac{a'b}{bb'} = \frac{a'}{b'} \]
    То есть чтобы перейти от первой дроби ко второй, надо домножить числитель и знаменатель на $b'$, а потом сократить на $b$. Таким образом, результат сложения или умножения не поменяется.
    
    Коммутативность и ассоциативность сложения очевидны в случае одинаковых знаменателей, а общий случай сводится к тому, чтобы заменить дроби на эквивалентные с одинаковым знаменателем.
    Ясно, что $\frac{0}{1}$ служит нейтральным по сложению, а дробь $\frac{-a}{-b}$ противоположна $\frac{a}{b}$.
    
    Для умножения очевидны коммутативность и ассоциативность, 
    а для проверки дистрибутивности также удобно записать складываемые дроби в виде с одинаковым знаменателем:
    \[ (\frac{a_1}{b} + \frac{a_2}{b})\frac{a'}{b'} = \frac{a_1 + a_2}{b}\frac{a'}{b'} 
    = \frac{a_1a' + a_2a'}{bb'} = \frac{a_1}{b}\frac{a'}{b'} + \frac{a_2}{b}\frac{a'}{b'} \]
    Дробь $\frac{1}{1}$ является единичным элементом, и если $b \neq 0$, то обратный к $\frac{a}{b}$ - это $\frac{b}{a}$.
\end{proof}

\begin{notice}
    $R \to Q(R)$ -- инъективный гомоморфизм колец, переводящий $r \in R$ в дробь $\frac{r}{1}$. Таким образом, $R$ является подкольцом $Q(R)$.
\end{notice}

\textbf{Примеры:}
\begin{enumerate}
    \item $Q(\mathbb{Z}) = \mathbb{Q}$
    \item Для кольца многочленов $K[X]$ это $Q(K[X]) = K(X)$ -- поле дробно-рациональных функций от одной переменной над $K$.
\end{enumerate}