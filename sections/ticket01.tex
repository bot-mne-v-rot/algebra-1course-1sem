\section{Алгебраически замкнутые поля, формулировка основной теоремы алгебры}

\begin{conj}
    Поле $K$ --- алгебраически замкнуто, если любой многочлен $f \in K[x]$
    положительной степени имеет корень в $K$.
    \textbf{Пример:} Поле комплексных чисел $\mathbb{C}$
\end{conj}

\begin{theorem-non}
    Если $K$ алгебраически замкнуто, то $f \in K[x]$ неприводим $\Longleftrightarrow deg f = 1$
    \begin{proof}
        Пусть это не так. \\
        $deg f \geqslant 2.$ Тогда $f(a) = 0$ для некоторого $a \in K$, откуда $(x - a) | f$ (по т. Безу),
        то есть $f$ --- приводим. Противоречие.
    \end{proof}
\end{theorem-non}

Таким образом каноническое разложение имеет вид (в алгебраически замкнутом поле $K$): \\
$f = c \prod_{i = 1}^{l} (x - a_i)^{n_i}$, где $c \in K^*,\; a_i$ --- различные корни многочлена,
$n_i$ --- натуральные степени. \\
При этом $n_i$ совпадает с кратностью корня $a_i$, а $c$ --- старший коэфициент.
