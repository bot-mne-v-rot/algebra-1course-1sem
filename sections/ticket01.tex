% !TEX root = ../LinalColloc02.tex
\section{Алгебраически замкнутые поля, формулировка основной теоремы алгебры}

\begin{conj}
    Поле $K$ называется алгебраически замкнутым, если любой многочлен $f \in K[X]$
    положительной степени имеет корень в $K$.
\end{conj}

\textbf{Теорема} (основная теорема алгебры). 
\textit{Поле $\mathbb{C}$ алгебраически замкнуто.}

\vspace{5mm}

\begin{theorem-non}
    Если $K$ алгебраически замкнуто, то $f \in K[X]$ неприводим $\Longleftrightarrow deg f = 1$
    \begin{proof}
        Пусть $deg f \geqslant 2.$ Тогда $f(a) = 0$ для некоторого $a \in K$, откуда $(x - a) | f$ (по т. Безу),
        то есть $f$ --- приводим. Противоречие.
    \end{proof}
\end{theorem-non}

Таким образом, каноническое разложение в алгебраически замкнутом поле $K$ имеет вид: 
\[ f = c \prod_{i = 1}^{l} (x - a_i)^{n_i}, \]
 где $c \in K^*,\; a_i$ --- различные корни многочлена,
$n_i$ --- натуральные степени. \\
При этом $n_i$ совпадает с кратностью корня $a_i$, а $c$ --- старший коэфициент.
